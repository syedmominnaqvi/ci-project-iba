\documentclass[conference]{IEEEtran}
\IEEEoverridecommandlockouts
% The preceding line is only needed to identify funding in the first footnote. If that is unneeded, please comment it out.
\usepackage{cite}
\usepackage{amsmath,amssymb,amsfonts}
\usepackage{algorithmic}
\usepackage{graphicx}
\usepackage{textcomp}
\usepackage{tikz}
\usetikzlibrary{arrows.meta,positioning}
\usepackage{xcolor}
\setlength{\parskip}{6pt}
\def\BibTeX{{\rm B\kern-.05em{\sc i\kern-.025em b}\kern-.08em
    T\kern-.1667em\lower.7ex\hbox{E}\kern-.125emX}}
\begin{document}

\title{Genetic Algorithm-Based Approach for PHI/PII Detection in Unstructured Text\\
}

\author{\IEEEauthorblockN{Syed Momin Naqvi}
\IEEEauthorblockA{\textit{Department of Computer Science} \\
\textit{Institute of Business Administration}\\
Karachi, Pakistan \\
syedmominnaqvi@gmail.com}
\and
\IEEEauthorblockN{Dr. Asma Sanam Larik}
\IEEEauthorblockA{\textit{Department of Computer Science} \\
\textit{Institute of Business Administration}\\
Karachi, Pakistan \\
alarik@iba.edu.pk}
\and
\IEEEauthorblockN{Rehmat Gul}
\IEEEauthorblockA{\textit{Department of Computer Science} \\
\textit{Institute of Business Administration}\\
Karachi, Pakistan \\
<todooooo>}
}

\maketitle

\begin{abstract}
This research presents a novel genetic algorithm approach for detecting Personal Health Information (PHI) and Personally Identifiable Information (PII) in unstructured text data. Traditional methods rely on rule-based systems or supervised learning that require extensive labeled data, which can be impractical in sensitive domains. Our approach evolves detection patterns using genetic algorithms with specialized operators designed for text pattern recognition. The system employs a unique chromosome representation where each gene corresponds to a detection pattern targeting specific types of sensitive information. We developed custom genetic operators for crossover and mutation that respect the semantic structure of text patterns. The fitness function balances precision, recall, and pattern complexity to evolve effective yet generalizable solutions. The system was evaluated on a diverse corpus of documents containing PHI/PII elements, achieving competitive performance compared to industry standards such as rigid Regex patterns. Results demonstrate the adaptability of evolutionary algorithms for sensitive information detection without requiring extensive training data, with particularly strong performance in detecting names, phone numbers, emails, and SSNs. The approach shows promising applications in healthcare, financial, and other domains where protecting sensitive information is critical.
\end{abstract}

\begin{IEEEkeywords}
evolutionary, PHI, PII, detection, genetic
\end{IEEEkeywords}

\section{Introduction}
This document is a model and instructions for \LaTeX.
Please observe the conference page limits.

\section{\textbf{Proposed Methodology}}

\subsection{\textbf{System Architecture Overview}}

Our PHI/PII detection system is designed as a robust, multi-component framework that integrates advanced document processing, evolutionary computation, pattern recognition, and performance evaluation to accurately detect and protect sensitive information. The system workflow is structured to ensure scalability, adaptability, and high detection accuracy across diverse document types and data formats.
\begin{enumerate}
\item Document Processing Pipeline: Handles various document formats (TXT, PDF, DOC, etc.) and extracts plain text for analysis.

\item Genetic Algorithm Engine: Implements the evolutionary approach with custom representations and operators.

\item  Pattern Matching Subsystem: Applies evolved detection patterns to identify PHI/PII in text.

\item Evaluation and Reporting: Assesses detection performance and generates detailed reports.
\end{enumerate}

The system workflow begins with document ingestion, where files are processed to extract plain text content. This content is then analyzed using the genetic algorithm, which evolves effective detection patterns. The evolved patterns are applied to identify PHI/PII instances, and results are evaluated against baseline approaches.
\paragraph{\textbf{Document Processing Pipeline}}

The first critical component is the Document Processing Pipeline, which serves as the entry point for all incoming data. This pipeline is engineered to handle a wide variety of document formats such as TXT, PDF, DOC, and other common file types, ensuring broad applicability in real-world scenarios.
\\
\textbf{Multi-Format Support}: The system processes several file formats including:
\begin{enumerate}
\item Plain text files (.txt, .csv, .json, .xml, .html, .md, .log)
\item PDF files with extraction via pdftotext command-line tool, PyPDF2, and pdfplumber
\item Microsoft Office documents when external libraries (textract or Tika) are available
\item Email files (.eml, .msg) with basic content extraction
\end{enumerate}

\textbf{Safety Measures}: Basic protections against processing errors:
\begin{enumerate}
\item File size limits (1MB maximum text size)
\item Page limits for PDFs (50 pages maximum)
\item Processing timeouts (30 seconds per PDF, 10 seconds for text files)
\item Exception handling
\item Comprehensive error logging
\end{enumerate}

\textbf{Text Extraction}: Format-specific extraction using appropriate libraries:
\begin{enumerate}
\item Direct reading for plain text formats
\item Multiple fallback mechanisms for PDFs (pdftotext → PyPDF2 → pdfplumber)
\item External dependencies for Office documents when available
\end{enumerate}

\textbf{Error Handling}: The pipeline includes:
\begin{enumerate}
\item Timeout management using signal handlers
\item Fallback processing for extraction failures
\item Graceful degradation when primary extraction methods fail
\end{enumerate}

\paragraph{\textbf{Genetic Algorithm Engine}}
The genetic algorithm engine forms the core intelligence component of our PHI/PII detection system. It implements a specialized evolutionary computation framework tailored for text pattern discovery and optimization.
\\
\\
Our genetic algorithm engine is implemented using the DEAP (Distributed Evolutionary Algorithms in Python) framework with custom components specifically designed for PHI/PII detection. The implementation balances computational efficiency with effective pattern discovery, providing a foundation for detecting sensitive information without requiring extensive labeled training data.
\\
Key technical aspects of the implementation include:
\begin{itemize}
\item \textbf{Creator Configuration}: Custom fitness class (FitnessMax) with weights (1.0, 1.0, -0.5)
\item \textbf{Toolbox Setup}: Registered operators for gene creation, chromosome formation, crossover, mutation, and selection
\item \textbf{Algorithm}: Uses eaSimple algorithm from DEAP with configured parameters
\item \textbf{Statistics Tracking}: Monitors fitness components throughout evolution
\end{itemize}

\paragraph{\textbf{Chromosome Design and Representation}}
We designed a specialized chromosome representation tailored to the PHI/PII detection task. Each chromosome contains multiple detection genes that collectively form a complete detection solution.

A detection gene contains the following chromosomes:
\begin{itemize}
\item \textbf{Pattern}: A regular expression pattern for matching text, selected from a predefined set of common PII patterns during initialization:
  \begin{itemize}
  \item Name patterns: \texttt{\textbackslash b[A-Z][a-z]+\textbackslash s+[A-Z][a-z]+\textbackslash b}
  \item Email patterns: \texttt{\textbackslash b[A-Za-z0-9.\%+-]+@[A-Za-z0-9.-]+\textbackslash.[A-Z|a-z]\{2,\}\textbackslash b}
  \item Phone patterns: \texttt{\textbackslash b\textbackslash d\{3\}[-.]?\textbackslash d\{3\}[-.]?\textbackslash d\{4\}\textbackslash b}
  \item SSN patterns: \texttt{\textbackslash b\textbackslash d\{3\}[-]?\textbackslash d\{2\}[-]?\textbackslash d\{4\}\textbackslash b}
  \end{itemize}
\item \textbf{PII Type}: The category of sensitive information the gene detects (NAME, EMAIL, PHONE, SSN, MEDICAL\_RECORD, DIAGNOSIS, MEDICATION, etc.)
\item \textbf{Context Window}: Integer parameter controlling surrounding text consideration (default: 5)
\item \textbf{Confidence Score}: Floating-point value (0.0-1.0) indicating detection reliability
\end{itemize}

This representation allows for flexible and expressive detection rules that can capture various types of sensitive information. The multi-gene structure enables a single chromosome to detect multiple PHI/PII categories simultaneously. Additionally, the variable-length chromosome design enables solutions to dynamically adjust complexity during evolution.

\paragraph{\textbf{Genetic Operators}}
The genetic operators utilized and defined in this algorithm are as follows:
\\
\\
\textbf{Parent Selection}
Before genetic operations can occur, parent individuals must be selected from the population. Our implementation uses tournament selection:

\begin{enumerate}
\item For each selection event, randomly sample a subgroup of individuals (tournament size: 3)
\item Compare fitness values within this subgroup
\item Select the individual with the highest fitness as a parent
\item Repeat the process to select the second parent
\end{enumerate}

Tournament selection provides several advantages for our PHI/PII detection system:
\begin{itemize}
\item It balances exploration and exploitation by allowing some less-fit individuals to participate
\item It maintains selection pressure toward higher fitness solutions
\item It doesn't require global ranking of all individuals, improving computational efficiency
\item Selection pressure can be adjusted through tournament size
\end{itemize}

After parent selection, the genetic operators are applied to create offspring for the next generation.

\textbf{Crossover Implementation}
Our crossover operator implements single-point exchange between parent chromosomes:

\begin{enumerate}
\item Random crossover points are independently selected in each parent chromosome
\item Gene sequences are exchanged between parent chromosomes to create two offspring
\item The operation respects gene boundaries, preserving the integrity of individual detection patterns
\item Crossover occurs with configurable probability (default: 0.7)
\end{enumerate}

This approach enables effective recombination of detection strategies while maintaining the semantic integrity of individual detection genes.

\textbf{{Mutation Implementation}}
Our mutation system operates at two distinct levels:
\\
\\
\textbf{Chromosome-Level Mutation} (controlled by mutation\_prob, default: 0.2):
\begin{itemize}
\item Applies gene-level mutations to individual genes
\item Adds new genes with 0.2 probability (gene addition)
\item Removes genes with 0.2 probability if chromosome has more than one gene (gene deletion)
\end{itemize}

\textbf{Gene-Level Mutation} (controlled by gene\_mutation\_rate, default: 0.3):
\begin{enumerate}
\item \textbf{Pattern Expansion}: Makes regex more general by:
   \begin{itemize}
   \item Expanding character classes (e.g., [A-Z] → [A-Za-z])
   \item Relaxing quantifiers (e.g., \{3\} → \{2,4\})
   \item Converting '+' to '*' for more flexible matching
   \end{itemize}
\item \textbf{Pattern Restriction}: Makes regex more specific by:
   \begin{itemize}
   \item Adding word boundaries (\textbackslash b)
   \item Tightening quantifiers (e.g., * → +)
   \item Converting character classes to specific characters
   \end{itemize}
\item \textbf{Context Window Adjustment}: Increases or decreases context window size by 1
\item \textbf{PII Type Change}: Shifts to an adjacent PII type in the predefined list
\item \textbf{Confidence Adjustment}: Applies random adjustment (±0.1) to confidence score
\end{enumerate}

This multi-level mutation strategy allows both micro-optimization of individual patterns and macro-level structural changes to the chromosome.

\textbf{Fitness Function Design}
We designed a multi-objective fitness function that balances three key aspects of detection performance:
\begin{enumerate}
\item \textbf{Precision}: TP / (TP + FP) - Proportion of detected items that are actual PHI/PII
\item \textbf{Recall}: TP / (TP + FN) - Proportion of actual PHI/PII items that are detected
\item \textbf{Complexity Penalty}: Calculated from number of genes and pattern complexity
\end{enumerate}

The fitness function is formulated as:
\begin{equation}
F(chromosome) = (precision, recall, -complexity)
\end{equation}

where the fitness values are weighted as (1.0, 1.0, -0.5) for precision, recall, and complexity respectively. The weight for complexity is kept negative in order to penalize the system for this negative attribute.
This creates a balanced optimization objective that favors accurate yet generalizable solutions.

For each chromosome evaluation:
\begin{itemize}
\item All genes are applied to input text to generate candidate matches
\item Matches are compared against known annotations
\item True positives, false positives, and false negatives are counted
\item Complexity is calculated based on gene count and pattern intricacy
\item The three fitness components are returned as a tuple
\end{itemize}

This multi-objective approach encourages the evolution of solutions that are both effective at detection and generalizable to new data.

\subsection{\textbf{Population Management and Algorithm Configuration}}

\subsubsection{Initialization Process}
The population is initialized with the following procedure:
\begin{enumerate}
\item Generate a population of individuals (default size: 50)
\item Each individual contains one chromosome
\item Initial chromosomes have a fixed number of genes (default: 5)
\item Genes are initialized with patterns selected from a predefined set of common PII patterns
\item Each gene is randomly assigned a PII type and default confidence (0.5)
\end{enumerate}

\subsection{\textbf{Selection Mechanism}}
The selection process uses tournament selection:
\begin{enumerate}
\item For each selection, randomly choose 3 individuals (tournament size)
\item Select the individual with the highest fitness from this group
\item Repeat the process to select parents for breeding
\end{enumerate}

\subsection{\textbf{Configuration Parameters}}
Our genetic algorithm uses the following configuration:
\begin{itemize}
\item Population Size: The size of the total population being considered
\item Generations: The number of generations for which the genetic algorithm is run
\item Selection Method: The method through which parents are selected for mating
\item Crossover Probability: Represents the probability of crossover between parents
\item Mutation Probability: The probability of minor mutations in the genes
\item Initial Chromosome Size: Represents the size of the initial population
\end{itemize}

These parameters were determined through preliminary experimentation to balance exploration of the solution space with convergence speed.

\section{\textbf{Experiments and Results}}
\subsection{\textbf{Dataset Description}}
We evaluated our genetic algorithm-based approach on a diverse corpus of medical and healthcare-related documents containing various types of PHI/PII. The dataset consisted of:

\begin{itemize}
\item Document Count: 1,489 documents from diverse healthcare domains
\item File Types: Multiple formats including:
\item Text files (.txt, .rtf, .csv): 421 documents
\item Email messages (.msg): 127 documents
\item PDF documents (.pdf): 612 documents
\item Microsoft Office documents (.doc, .docx, .xls, .xlsx): 329 documents
\item Content Variety: Documents spanned clinical records, medical forms, lab results, patient correspondence, and administrative communications
\item Text Volume: Text length varied significantly, from brief medical notes ($<$ 100 characters) to comprehensive medical records ($>$ 1 million characters)
\item Document Categories: Documents were organized into medical domains including:
\begin{itemize}
\item   Patient identification documents
\item   Medical service records
\item   Disease and condition reports
\item   Medication information
\item   Lab test results
\item   Vital signs records
\item   Medical consent forms
\item   Healthcare provider communications
\end{itemize}
\end{itemize}
This diverse collection provided a realistic representation of the varied document types encountered in healthcare environments, making it an ideal testbed for evaluating PHI/PII detection capabilities.

\subsection{\textbf{PHI/PII Content Characteristics}}
The dataset contained various types of sensitive information commonly found in healthcare documents:

\begin{enumerate}
\item Patient Names: Both full names and name components (first, middle, last)
\paragraph{Contextual Appearance Patterns:}
\begin{itemize}
\item Header information in medical reports and forms
\item Signature lines and authorization sections
\item Emergency contact listings with relationship indicators
\item Insurance beneficiary information and policy holder details
\item Appointment scheduling records and patient registration forms
\end{itemize}

\item Contact Information: Phone numbers and email addresses in various formats
\paragraph{Contextual Appearance Patterns:}
\begin{itemize}
\item Primary contact information in patient demographics
\item Emergency contact details with relationship specifications
\item Healthcare provider contact information and referral networks
\item Insurance company communication channels and claim correspondence
\item Appointment confirmation and reminder system integrations
\end{itemize}
\item Identification Numbers: Social Security Numbers (SSNs) and other identification codes
\item Medical Identifiers: Record numbers, patient IDs, and provider identifiers
\item Dates: Dates of service, birth dates, and admission/discharge dates
\item Addresses: Patient and facility addresses in various formats
\end{enumerate}
These PHI/PII elements appeared in different contexts and formats, creating a challenging detection environment that mirrors real-world healthcare documentation.

\subsection{\textbf{Baseline Methods}}
We compared our genetic algorithm approach against the baseline method:

Rule-based Pattern Matching: A traditional approach using fixed regular expressions for common PHI/PII patterns:
\begin{enumerate}
    \item \textbf{Names:} \texttt{[A-Z][a-z]+[\textbackslash s]+[A-Z][a-z]+}

    \item \textbf{Phone numbers:} \texttt{\textbackslash b\d\{3\}[-.]?\d\{3\}[-.]?\d\{4\}\textbackslash b}

    \item \textbf{Email addresses:} \texttt{\textbackslash b[A-Za-z0-9.\_\%\allowbreak +-]+@[A-Za-z0-9.-]+\textbackslash.[A-Za-z]\allowbreak \{2,\}\textbackslash b}

    \item \textbf{SSNs:} \texttt{\textbackslash b\d\{3\}-?\d\{2\}-?\d\{4\}\textbackslash b}

    \item \textbf{Dates:} \texttt{\textbackslash b\d\{1,2\}/\d\{1,2\}/\d\{2,4\}\textbackslash b}

    \item \textbf{Medical record numbers:} \texttt{\textbackslash bMRN:?\textbackslash s*\d+\textbackslash b}

\end{enumerate}


This baseline represents a common approach to PHI/PII detection: fixed rule-based patterns (representing traditional methods)

\subsection{\textbf{Genetic Algorithm Configuration}}
For our experimental evaluation, we configured the genetic algorithm with the following parameters:
\renewcommand{\arraystretch}{1.4}  % Increase row height spacing

\begin{table}[htbp]
\caption{Genetic Algorithm Configuration Parameters}
\centering
\begin{tabular}{|l|l|}
\hline
\textbf{Parameter} & \textbf{Value} \\
\hline
Population Size & 50 chromosomes \\
Generations & 100 \\
Selection Method & Tournament selection \\
Tournament size & 3\\
Crossover Probability & 0.7 \\
Mutation Probability & 0.3 \\
Initial Chromosome Size & 5 genes \\
Fitness Function Weights & (1.0, 1.0, -0.5) for precision, recall, and complexity \\
\hline
\end{tabular}
\label{tab:ga_config}
\end{table}

This configuration was determined through preliminary experiments to balance exploration of the pattern space with convergence speed. The increased mutation rate (0.3) was specifically chosen to promote better exploration of the pattern space.

\subsection{\textbf{Hardware Configuration}}
All experiments were conducted on a dedicated testing environment with the following specifications:

\begin{table}[htbp]
\caption{System Configuration for Experimental Setup}
\centering
\renewcommand{\arraystretch}{2.5}
\begin{tabular}{|l|l|}
\hline
\textbf{Component} & \textbf{Specification} \\
\hline
Processor & Apple M1 Pro \\
Cores & 8 \\
Performance cores & 6 \\
Efficiency cores & 2 \\
GPU & 14-core \\
Neural Engine & 16-core  \\
Memory & 32 GB \\
Storage & 512 GB SSD \\
Operating System & MacOS Sequoia 15.4.1 \\
Libraries & Python 3.11, DEAP, NumPy, Pandas \\
\hline
\end{tabular}
\label{tab:system-specs}
\end{table}

These hardware specifications were consistent across all experiments to ensure fair performance comparisons between the genetic algorithm approach and baseline methods. The processing time metrics reported in our results are specific to this configuration and may vary on different hardware.

\subsection{\textbf{Experimental Protocol}}
We conducted our experiments using the following protocol:
\begin{enumerate}
\item Document Processing: Documents were processed in batches of 5 files for efficient training and evaluation
\item Training Data Generation: Since labeled PHI/PII datasets are unavailable due to privacy concerns, we generated synthetic annotations using Rule-based pattern matching as the primary annotation source
\item Genetic Algorithm Training: \\
For each batch:
\begin{enumerate}
\item Initial population was randomly generated
\item Evolution proceeded for 100 generations
\item Best individual was selected based on combined fitness score
\end{enumerate}
\item Evaluation: \\
We assessed:
\begin{enumerate}
\item Detection performance (precision, recall, F1 score)
\item PHI/PII type coverage
\item Processing time and computational requirements
\item Convergence behavior through fitness curves
\end{enumerate}
\end{enumerate}

The experimental framework was designed to simulate real-world PHI/PII detection scenarios while maintaining rigorous evaluation standards.

\section{\textbf{Results}}
Our genetic algorithm successfully detected 33,711 instances of PHI/PII across 526 documents (35.3\% of the corpus). Table 1 presents the overall detection statistics.

\begin{table}[htbp]
\caption{Overall PHI/PII Detection Statistics}
\centering
\renewcommand{\arraystretch}{2.5}
\begin{tabular}{|l|r|}
\hline
\textbf{Metric} & \textbf{Value} \\
\hline
Total documents processed & 1,489 \\
Documents containing PHI/PII & 526 (35.3\%) \\
Total PHI/PII instances detected & 33,711 \\
Total processing time & 2,430.7 seconds (40.5 minutes) \\
Average processing time per document & 1.63 seconds \\
\hline
\end{tabular}
\label{tab:overall-stats}
\end{table}

\subsection{\textbf{PHI/PII Type Distribution}}
The genetic algorithm detected four main categories of PHI/PII. Table 4 shows the performance breakdown by PHI/PII type.

Table 2: Detection Performance by PHI/PII Type

\begin{table}[htbp]
\caption{PHI/PII Detection Performance by Type}
\centering
\renewcommand{\arraystretch}{2.5}
\begin{tabular}{|l|r|r|r|r|r|}
\hline
\textbf{PHI/PII Type} & \textbf{Count} & \textbf{\%} & \textbf{Precision} & \textbf{Recall} & \textbf{F1 Score} \\
\hline
NAME & 15,267 & 45.3\% & 0.81 & 0.84 & 0.82 \\
PHONE & 8,595 & 25.5\% & 0.94 & 0.89 & 0.91 \\
EMAIL & 5,312 & 15.8\% & 0.96 & 0.92 & 0.94 \\
SSN & 4,537 & 13.5\% & 0.98 & 0.87 & 0.92 \\
\textbf{Overall} & \textbf{33,711} & \textbf{100\%} & \textbf{0.89} & \textbf{0.87} & \textbf{0.88} \\
\hline
\end{tabular}
\label{tab:phi-pii-performance}
\end{table}

The results show that our genetic algorithm achieved strong overall performance (F1 score of 0.88), with particularly high precision for structured patterns like SSNs (0.98) and emails (0.96). Names represented the largest category of detected PHI/PII (45.3\%), followed by phone numbers (25.5\%).








\section{Prepare Your Paper Before Styling}
Before you begin to format your paper, first write and save the content as a
separate text file. Complete all content and organizational editing before
formatting. Please note sections \ref{AA}--\ref{SCM} below for more information on
proofreading, spelling and grammar.

Keep your text and graphic files separate until after the text has been
formatted and styled. Do not number text heads---{\LaTeX} will do that
for you.

\subsection{Abbreviations and Acronyms}\label{AA}
Define abbreviations and acronyms the first time they are used in the text,
even after they have been defined in the abstract. Abbreviations such as
IEEE, SI, MKS, CGS, ac, dc, and rms do not have to be defined. Do not use
abbreviations in the title or heads unless they are unavoidable.

\subsection{Units}
\begin{itemize}
\item Use either SI (MKS) or CGS as primary units. (SI units are encouraged.) English units may be used as secondary units (in parentheses). An exception would be the use of English units as identifiers in trade, such as ``3.5-inch disk drive''.
\item Avoid combining SI and CGS units, such as current in amperes and magnetic field in oersteds. This often leads to confusion because equations do not balance dimensionally. If you must use mixed units, clearly state the units for each quantity that you use in an equation.
\item Do not mix complete spellings and abbreviations of units: ``Wb/m\textsuperscript{2}'' or ``webers per square meter'', not ``webers/m\textsuperscript{2}''. Spell out units when they appear in text: ``. . . a few henries'', not ``. . . a few H''.
\item Use a zero before decimal points: ``0.25'', not ``.25''. Use ``cm\textsuperscript{3}'', not ``cc''.)
\end{itemize}

\subsection{Equations}
Number equations consecutively. To make your
equations more compact, you may use the solidus (~/~), the exp function, or
appropriate exponents. Italicize Roman symbols for quantities and variables,
but not Greek symbols. Use a long dash rather than a hyphen for a minus
sign. Punctuate equations with commas or periods when they are part of a
sentence, as in:
\begin{equation}
a+b=\gamma\label{eq}
\end{equation}

Be sure that the
symbols in your equation have been defined before or immediately following
the equation. Use ``\eqref{eq}'', not ``Eq.~\eqref{eq}'' or ``equation \eqref{eq}'', except at
the beginning of a sentence: ``Equation \eqref{eq} is . . .''

\subsection{\LaTeX-Specific Advice}

Please use ``soft'' (e.g., \verb|\eqref{Eq}|) cross references instead
of ``hard'' references (e.g., \verb|(1)|). That will make it possible
to combine sections, add equations, or change the order of figures or
citations without having to go through the file line by line.

Please don't use the \verb|{eqnarray}| equation environment. Use
\verb|{align}| or \verb|{IEEEeqnarray}| instead. The \verb|{eqnarray}|
environment leaves unsightly spaces around relation symbols.

Please note that the \verb|{subequations}| environment in {\LaTeX}
will increment the main equation counter even when there are no
equation numbers displayed. If you forget that, you might write an
article in which the equation numbers skip from (17) to (20), causing
the copy editors to wonder if you've discovered a new method of
counting.

{\BibTeX} does not work by magic. It doesn't get the bibliographic
data from thin air but from .bib files. If you use {\BibTeX} to produce a
bibliography you must send the .bib files.

{\LaTeX} can't read your mind. If you assign the same label to a
subsubsection and a table, you might find that Table I has been cross
referenced as Table IV-B3.

{\LaTeX} does not have precognitive abilities. If you put a
\verb|\label| command before the command that updates the counter it's
supposed to be using, the label will pick up the last counter to be
cross referenced instead. In particular, a \verb|\label| command
should not go before the caption of a figure or a table.

Do not use \verb|\nonumber| inside the \verb|{array}| environment. It
will not stop equation numbers inside \verb|{array}| (there won't be
any anyway) and it might stop a wanted equation number in the
surrounding equation.

\subsection{Some Common Mistakes}\label{SCM}
\begin{itemize}
\item The word ``data'' is plural, not singular.
\item The subscript for the permeability of vacuum $\mu_{0}$, and other common scientific constants, is zero with subscript formatting, not a lowercase letter ``o''.
\item In American English, commas, semicolons, periods, question and exclamation marks are located within quotation marks only when a complete thought or name is cited, such as a title or full quotation. When quotation marks are used, instead of a bold or italic typeface, to highlight a word or phrase, punctuation should appear outside of the quotation marks. A parenthetical phrase or statement at the end of a sentence is punctuated outside of the closing parenthesis (like this). (A parenthetical sentence is punctuated within the parentheses.)
\item A graph within a graph is an ``inset'', not an ``insert''. The word alternatively is preferred to the word ``alternately'' (unless you really mean something that alternates).
\item Do not use the word ``essentially'' to mean ``approximately'' or ``effectively''.
\item In your paper title, if the words ``that uses'' can accurately replace the word ``using'', capitalize the ``u''; if not, keep using lower-cased.
\item Be aware of the different meanings of the homophones ``affect'' and ``effect'', ``complement'' and ``compliment'', ``discreet'' and ``discrete'', ``principal'' and ``principle''.
\item Do not confuse ``imply'' and ``infer''.
\item The prefix ``non'' is not a word; it should be joined to the word it modifies, usually without a hyphen.
\item There is no period after the ``et'' in the Latin abbreviation ``et al.''.
\item The abbreviation ``i.e.'' means ``that is'', and the abbreviation ``e.g.'' means ``for example''.
\end{itemize}
An excellent style manual for science writers is \cite{b7}.

\subsection{Authors and Affiliations}
\textbf{The class file is designed for, but not limited to, six authors.} A
minimum of one author is required for all conference articles. Author names
should be listed starting from left to right and then moving down to the
next line. This is the author sequence that will be used in future citations
and by indexing services. Names should not be listed in columns nor group by
affiliation. Please keep your affiliations as succinct as possible (for
example, do not differentiate among departments of the same organization).

\subsection{Identify the Headings}
Headings, or heads, are organizational devices that guide the reader through
your paper. There are two types: component heads and text heads.

Component heads identify the different components of your paper and are not
topically subordinate to each other. Examples include Acknowledgments and
References and, for these, the correct style to use is ``Heading 5''. Use
``figure caption'' for your Figure captions, and ``table head'' for your
table title. Run-in heads, such as ``Abstract'', will require you to apply a
style (in this case, italic) in addition to the style provided by the drop
down menu to differentiate the head from the text.

Text heads organize the topics on a relational, hierarchical basis. For
example, the paper title is the primary text head because all subsequent
material relates and elaborates on this one topic. If there are two or more
sub-topics, the next level head (uppercase Roman numerals) should be used
and, conversely, if there are not at least two sub-topics, then no subheads
should be introduced.

\subsection{Figures and Tables}
\paragraph{Positioning Figures and Tables} Place figures and tables at the top and
bottom of columns. Avoid placing them in the middle of columns. Large
figures and tables may span across both columns. Figure captions should be
below the figures; table heads should appear above the tables. Insert
figures and tables after they are cited in the text. Use the abbreviation
``Fig.~\ref{fig}'', even at the beginning of a sentence.

\begin{table}[htbp]
\caption{Table Type Styles}
\begin{center}
\begin{tabular}{|c|c|c|c|}
\hline
\textbf{Table}&\multicolumn{3}{|c|}{\textbf{Table Column Head}} \\
\cline{2-4}
\textbf{Head} & \textbf{\textit{Table column subhead}}& \textbf{\textit{Subhead}}& \textbf{\textit{Subhead}} \\
\hline
copy& More table copy$^{\mathrm{a}}$& &  \\
\hline
\multicolumn{4}{l}{$^{\mathrm{a}}$Sample of a Table footnote.}
\end{tabular}
\label{tab1}
\end{center}
\end{table}

\begin{figure}[htbp]
\centerline{\includegraphics{fig1.png}}
\caption{Example of a figure caption.}
\label{fig}
\end{figure}

Figure Labels: Use 8 point Times New Roman for Figure labels. Use words
rather than symbols or abbreviations when writing Figure axis labels to
avoid confusing the reader. As an example, write the quantity
``Magnetization'', or ``Magnetization, M'', not just ``M''. If including
units in the label, present them within parentheses. Do not label axes only
with units. In the example, write ``Magnetization (A/m)'' or ``Magnetization
\{A[m(1)]\}'', not just ``A/m''. Do not label axes with a ratio of
quantities and units. For example, write ``Temperature (K)'', not
``Temperature/K''.

\section*{Acknowledgment}

The preferred spelling of the word ``acknowledgment'' in America is without
an ``e'' after the ``g''. Avoid the stilted expression ``one of us (R. B.
G.) thanks $\ldots$''. Instead, try ``R. B. G. thanks$\ldots$''. Put sponsor
acknowledgments in the unnumbered footnote on the first page.

\section*{References}

Please number citations consecutively within brackets \cite{b1}. The
sentence punctuation follows the bracket \cite{b2}. Refer simply to the reference
number, as in \cite{b3}---do not use ``Ref. \cite{b3}'' or ``reference \cite{b3}'' except at
the beginning of a sentence: ``Reference \cite{b3} was the first $\ldots$''

Number footnotes separately in superscripts. Place the actual footnote at
the bottom of the column in which it was cited. Do not put footnotes in the
abstract or reference list. Use letters for table footnotes.

Unless there are six authors or more give all authors' names; do not use
``et al.''. Papers that have not been published, even if they have been
submitted for publication, should be cited as ``unpublished'' \cite{b4}. Papers
that have been accepted for publication should be cited as ``in press'' \cite{b5}.
Capitalize only the first word in a paper title, except for proper nouns and
element symbols.

For papers published in translation journals, please give the English
citation first, followed by the original foreign-language citation \cite{b6}.

\begin{thebibliography}{00}
\bibitem{b1} G. Eason, B. Noble, and I. N. Sneddon, ``On certain integrals of Lipschitz-Hankel type involving products of Bessel functions,'' Phil. Trans. Roy. Soc. London, vol. A247, pp. 529--551, April 1955.
\bibitem{b2} J. Clerk Maxwell, A Treatise on Electricity and Magnetism, 3rd ed., vol. 2. Oxford: Clarendon, 1892, pp.68--73.
\bibitem{b3} I. S. Jacobs and C. P. Bean, ``Fine particles, thin films and exchange anisotropy,'' in Magnetism, vol. III, G. T. Rado and H. Suhl, Eds. New York: Academic, 1963, pp. 271--350.
\bibitem{b4} K. Elissa, ``Title of paper if known,'' unpublished.
\bibitem{b5} R. Nicole, ``Title of paper with only first word capitalized,'' J. Name Stand. Abbrev., in press.
\bibitem{b6} Y. Yorozu, M. Hirano, K. Oka, and Y. Tagawa, ``Electron spectroscopy studies on magneto-optical media and plastic substrate interface,'' IEEE Transl. J. Magn. Japan, vol. 2, pp. 740--741, August 1987 [Digests 9th Annual Conf. Magnetics Japan, p. 301, 1982].
\bibitem{b7} M. Young, The Technical Writer's Handbook. Mill Valley, CA: University Science, 1989.
\end{thebibliography}
\vspace{12pt}
\color{red}
IEEE conference templates contain guidance text for composing and formatting conference papers. Please ensure that all template text is removed from your conference paper prior to submission to the conference. Failure to remove the template text from your paper may result in your paper not being published.

\end{document}
